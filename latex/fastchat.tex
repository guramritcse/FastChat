%% Generated by Sphinx.
\def\sphinxdocclass{report}
\documentclass[letterpaper,10pt,english]{sphinxmanual}
\ifdefined\pdfpxdimen
   \let\sphinxpxdimen\pdfpxdimen\else\newdimen\sphinxpxdimen
\fi \sphinxpxdimen=.75bp\relax
\ifdefined\pdfimageresolution
    \pdfimageresolution= \numexpr \dimexpr1in\relax/\sphinxpxdimen\relax
\fi
%% let collapsible pdf bookmarks panel have high depth per default
\PassOptionsToPackage{bookmarksdepth=5}{hyperref}

\PassOptionsToPackage{warn}{textcomp}
\usepackage[utf8]{inputenc}
\ifdefined\DeclareUnicodeCharacter
% support both utf8 and utf8x syntaxes
  \ifdefined\DeclareUnicodeCharacterAsOptional
    \def\sphinxDUC#1{\DeclareUnicodeCharacter{"#1}}
  \else
    \let\sphinxDUC\DeclareUnicodeCharacter
  \fi
  \sphinxDUC{00A0}{\nobreakspace}
  \sphinxDUC{2500}{\sphinxunichar{2500}}
  \sphinxDUC{2502}{\sphinxunichar{2502}}
  \sphinxDUC{2514}{\sphinxunichar{2514}}
  \sphinxDUC{251C}{\sphinxunichar{251C}}
  \sphinxDUC{2572}{\textbackslash}
\fi
\usepackage{cmap}
\usepackage[T1]{fontenc}
\usepackage{amsmath,amssymb,amstext}
\usepackage{babel}



\usepackage{tgtermes}
\usepackage{tgheros}
\renewcommand{\ttdefault}{txtt}



\usepackage[Bjarne]{fncychap}
\usepackage{sphinx}

\fvset{fontsize=auto}
\usepackage{geometry}


% Include hyperref last.
\usepackage{hyperref}
% Fix anchor placement for figures with captions.
\usepackage{hypcap}% it must be loaded after hyperref.
% Set up styles of URL: it should be placed after hyperref.
\urlstyle{same}

\addto\captionsenglish{\renewcommand{\contentsname}{Contents:}}

\usepackage{sphinxmessages}
\setcounter{tocdepth}{1}



\title{FastChat}
\date{Nov 25, 2022}
\release{}
\author{Thrice as Nice}
\newcommand{\sphinxlogo}{\vbox{}}
\renewcommand{\releasename}{}
\makeindex
\begin{document}

\pagestyle{empty}
\sphinxmaketitle
\pagestyle{plain}
\sphinxtableofcontents
\pagestyle{normal}
\phantomsection\label{\detokenize{index::doc}}



\chapter{FastChat}
\label{\detokenize{modules:fastchat}}\label{\detokenize{modules::doc}}

\section{client module}
\label{\detokenize{client:module-client}}\label{\detokenize{client:client-module}}\label{\detokenize{client::doc}}\index{module@\spxentry{module}!client@\spxentry{client}}\index{client@\spxentry{client}!module@\spxentry{module}}\index{main() (in module client)@\spxentry{main()}\spxextra{in module client}}

\begin{fulllineitems}
\phantomsection\label{\detokenize{client:client.main}}\pysiglinewithargsret{\sphinxcode{\sphinxupquote{client.}}\sphinxbfcode{\sphinxupquote{main}}}{}{}
\sphinxAtStartPar
This is the main function
Firstly, a socket connection is initiated between loadbalancer and client, when the user logs in, a server is alloted to it.
Client disconnects its connection with the loadbalancer and a socket connection is initiated between the client and the server alloted to it
All the messages received by the client while it was offline are displayed and receiving and sending threads are initiated.

\end{fulllineitems}

\index{receiving\_func() (in module client)@\spxentry{receiving\_func()}\spxextra{in module client}}

\begin{fulllineitems}
\phantomsection\label{\detokenize{client:client.receiving_func}}\pysiglinewithargsret{\sphinxcode{\sphinxupquote{client.}}\sphinxbfcode{\sphinxupquote{receiving\_func}}}{}{}
\sphinxAtStartPar
It is thread for receiving information from the server
Depending on the code received from the server, the parameters will be changed accordingly.
The codes received from the server have the following meaning:
\begin{quote}

\begin{DUlineblock}{0em}
\item[] 1. c: Receive another code from the server
\item[] 2. y: Message has been sent successfully
\item[] 3. n: Unable to send direct message
\item[] 4. l: Group is full and hence no further member can be added
\item[] 5. t: User exists already in the group
\item[] 6. e: Receiving public key of receiver
\item[] 7. k: Receiving private key of group
\item[] 8. p: Decrypting group private key using private key of user
\item[] 9. s: Server is going to send the usernames of all the members of the group to the admin
\item[] 10. u: Have to encrypt the message of sender using the public key of receiver sent by server and have to send it back to the server
\item[] 11. g: Have to encrypt the message of sender using the public key of group sent by server and send the encrypted message back to server
\item[] 12. a: Received an image file
\item[] 13. b: Sent an image file
\item[] 14. q: Client has logged out
\end{DUlineblock}
\end{quote}

\end{fulllineitems}

\index{user\_interface() (in module client)@\spxentry{user\_interface()}\spxextra{in module client}}

\begin{fulllineitems}
\phantomsection\label{\detokenize{client:client.user_interface}}\pysiglinewithargsret{\sphinxcode{\sphinxupquote{client.}}\sphinxbfcode{\sphinxupquote{user\_interface}}}{\emph{\DUrole{n}{display\_menu}\DUrole{o}{=}\DUrole{default_value}{0}}}{}
\sphinxAtStartPar
Function to display the user interface
It is the thread for sending information to the server
Different user intefaces are displayed as per the value of the display\_menu parameter
The parameter display\_menu can take the following values and the following options are displayed as per the value of the display\_menu:
\begin{quote}

\begin{DUlineblock}{0em}
\item[] 1. display\_menu = 0 : Main Menu
\end{DUlineblock}
\begin{quote}

\begin{DUlineblock}{0em}
\item[] 1. g: for managing groups
\item[] 2. b: to send group message
\item[] 3. d: to send direct message
\item[] 4. l: to logout
\end{DUlineblock}
\end{quote}

\begin{DUlineblock}{0em}
\item[] 2. display\_menu = 1 : Group Settings
\end{DUlineblock}
\begin{quote}

\begin{DUlineblock}{0em}
\item[] 1. n: to create a new group
\item[] 2. m: to manage an existing group
\item[] 3. q: to go to previous menu
\end{DUlineblock}
\end{quote}

\begin{DUlineblock}{0em}
\item[] 3. display\_menu = 2 : Manage Existing Group
\end{DUlineblock}
\begin{quote}

\begin{DUlineblock}{0em}
\item[] 1. a: to add a new member
\item[] 2. r: to remove a member
\item[] 3. s: to see all members in the group
\item[] 4. q: to go to previous menu
\end{DUlineblock}
\end{quote}

\begin{DUlineblock}{0em}
\item[] 4. display\_menu = 3 : Group message
\end{DUlineblock}
\begin{quote}

\begin{DUlineblock}{0em}
\item[] 1. t: to type a message
\item[] 2. i: to send an image or text file
\item[] 3: q: to go to previous menu
\end{DUlineblock}
\end{quote}

\begin{DUlineblock}{0em}
\item[] 5. display\_menu = 4 : Direct message
\end{DUlineblock}
\begin{quote}

\begin{DUlineblock}{0em}
\item[] 1. t: to type a message
\item[] 2. i: to send an image or text file
\item[] 3. q: to go to previous menu
\end{DUlineblock}
\end{quote}
\end{quote}
\begin{quote}\begin{description}
\item[{Parameters}] \leavevmode
\sphinxAtStartPar
\sphinxstyleliteralstrong{\sphinxupquote{display\_menu}} (\sphinxstyleliteralemphasis{\sphinxupquote{int}}) \textendash{} contains information about which display menu is to be shown

\end{description}\end{quote}

\end{fulllineitems}



\section{loadbalancer module}
\label{\detokenize{loadbalancer:module-loadbalancer}}\label{\detokenize{loadbalancer:loadbalancer-module}}\label{\detokenize{loadbalancer::doc}}\index{module@\spxentry{module}!loadbalancer@\spxentry{loadbalancer}}\index{loadbalancer@\spxentry{loadbalancer}!module@\spxentry{module}}\index{clientthread() (in module loadbalancer)@\spxentry{clientthread()}\spxextra{in module loadbalancer}}

\begin{fulllineitems}
\phantomsection\label{\detokenize{loadbalancer:loadbalancer.clientthread}}\pysiglinewithargsret{\sphinxcode{\sphinxupquote{loadbalancer.}}\sphinxbfcode{\sphinxupquote{clientthread}}}{\emph{\DUrole{n}{conn}}, \emph{\DUrole{n}{addr}}}{}
\sphinxAtStartPar
This is the receiving thraed of the loadbalancer which receives code and performs the respective function
The loadbalancer receives the following codes which have the following meanings:
\begin{quote}

\begin{DUlineblock}{0em}
\item[] 1. s: The following can be received after it which have the following meanings
\end{DUlineblock}
\begin{quote}

\begin{DUlineblock}{0em}
\item[] 1. ci: Send the public key of the user
\item[] 2. cg: Send the public key of the group
\item[] 3. cs: Send the ip address and port number of the server to which the user is connected to
\item[] 4. cl: The user has logged out and thus should be deleted from the loadbalancer dictionary
\item[] 5. ag: Store the public key of the group in a dictionary of the load balancer
\end{DUlineblock}
\end{quote}

\begin{DUlineblock}{0em}
\item[] 2. c: Perform sign\sphinxhyphen{}up and login of the user and stores its encrypted private key and public key in a dictionary
\end{DUlineblock}
\end{quote}
\begin{quote}\begin{description}
\item[{Parameters}] \leavevmode\begin{itemize}
\item {} 
\sphinxAtStartPar
\sphinxstyleliteralstrong{\sphinxupquote{conn}} (\sphinxstyleliteralemphasis{\sphinxupquote{socket.socket}}) \textendash{} It is the socket object of the client/server which wants to establish socket connection to load balancer

\item {} 
\sphinxAtStartPar
\sphinxstyleliteralstrong{\sphinxupquote{addr}} (\sphinxstyleliteralemphasis{\sphinxupquote{tuple}}) \textendash{} It is the address tuple of the client/server (ip,port) which wants to establish connection with the load balancer

\end{itemize}

\end{description}\end{quote}

\end{fulllineitems}

\index{least\_connection() (in module loadbalancer)@\spxentry{least\_connection()}\spxextra{in module loadbalancer}}

\begin{fulllineitems}
\phantomsection\label{\detokenize{loadbalancer:loadbalancer.least_connection}}\pysiglinewithargsret{\sphinxcode{\sphinxupquote{loadbalancer.}}\sphinxbfcode{\sphinxupquote{least\_connection}}}{\emph{\DUrole{n}{server\_list}}}{}
\sphinxAtStartPar
Returns the port number and ip address of the server with the mminimum number of clients
\begin{quote}\begin{description}
\item[{Parameters}] \leavevmode
\sphinxAtStartPar
\sphinxstyleliteralstrong{\sphinxupquote{server\_list}} (\sphinxstyleliteralemphasis{\sphinxupquote{list}}\sphinxstyleliteralemphasis{\sphinxupquote{{[}}}\sphinxstyleliteralemphasis{\sphinxupquote{int}}\sphinxstyleliteralemphasis{\sphinxupquote{{]}}}) \textendash{} List of all servers

\item[{Returns}] \leavevmode
\sphinxAtStartPar
returns the port number and ip address of the server with minium number of clients

\item[{Return type}] \leavevmode
\sphinxAtStartPar
tuple

\end{description}\end{quote}

\end{fulllineitems}

\index{main() (in module loadbalancer)@\spxentry{main()}\spxextra{in module loadbalancer}}

\begin{fulllineitems}
\phantomsection\label{\detokenize{loadbalancer:loadbalancer.main}}\pysiglinewithargsret{\sphinxcode{\sphinxupquote{loadbalancer.}}\sphinxbfcode{\sphinxupquote{main}}}{}{}
\sphinxAtStartPar
Accepts a connection request and stores two parameters,conn which is a socket object for that user, and addr which contains the IP address of the client that just connected
Additionally, the databases for storing the credentials of all the clients, information of groups, individual messages for offline users and encrypted private keys of clients and their public keys are stored

\end{fulllineitems}

\index{round\_robin() (in module loadbalancer)@\spxentry{round\_robin()}\spxextra{in module loadbalancer}}

\begin{fulllineitems}
\phantomsection\label{\detokenize{loadbalancer:loadbalancer.round_robin}}\pysiglinewithargsret{\sphinxcode{\sphinxupquote{loadbalancer.}}\sphinxbfcode{\sphinxupquote{round\_robin}}}{\emph{\DUrole{n}{iter}}}{}
\sphinxAtStartPar
Implements round robin algorithm to return the pointer to the next server to which the incoming client would get connected to
\begin{quote}\begin{description}
\item[{Parameters}] \leavevmode
\sphinxAtStartPar
\sphinxstyleliteralstrong{\sphinxupquote{iter}} (\sphinxstyleliteralemphasis{\sphinxupquote{itertools.cycle}}) \textendash{} it points to the server to which the next incoing client will be connected to

\end{description}\end{quote}

\end{fulllineitems}

\index{select\_server() (in module loadbalancer)@\spxentry{select\_server()}\spxextra{in module loadbalancer}}

\begin{fulllineitems}
\phantomsection\label{\detokenize{loadbalancer:loadbalancer.select_server}}\pysiglinewithargsret{\sphinxcode{\sphinxupquote{loadbalancer.}}\sphinxbfcode{\sphinxupquote{select\_server}}}{\emph{\DUrole{n}{server\_list}}, \emph{\DUrole{n}{algorithm}}}{}
\sphinxAtStartPar
This function returns the tuple object of the server to which the incoming client would get connected to
\begin{quote}\begin{description}
\item[{Parameters}] \leavevmode\begin{itemize}
\item {} 
\sphinxAtStartPar
\sphinxstyleliteralstrong{\sphinxupquote{server\_list}} (\sphinxstyleliteralemphasis{\sphinxupquote{list of the tuples containing port and ip of each server}}) \textendash{} 

\item {} 
\sphinxAtStartPar
\sphinxstyleliteralstrong{\sphinxupquote{algorithm}} (\sphinxstyleliteralemphasis{\sphinxupquote{str}}) \textendash{} This is the algorithm which needs to be followed to select the server

\end{itemize}

\item[{Returns}] \leavevmode
\sphinxAtStartPar
Returns the tuple object (ip,address) of the server to which the incoming client will be connected

\item[{Return type}] \leavevmode
\sphinxAtStartPar
tuple containig ip and address of the server

\end{description}\end{quote}

\end{fulllineitems}



\section{server module}
\label{\detokenize{server:module-server}}\label{\detokenize{server:server-module}}\label{\detokenize{server::doc}}\index{module@\spxentry{module}!server@\spxentry{server}}\index{server@\spxentry{server}!module@\spxentry{module}}\index{clientthread() (in module server)@\spxentry{clientthread()}\spxextra{in module server}}

\begin{fulllineitems}
\phantomsection\label{\detokenize{server:server.clientthread}}\pysiglinewithargsret{\sphinxcode{\sphinxupquote{server.}}\sphinxbfcode{\sphinxupquote{clientthread}}}{\emph{\DUrole{n}{conn}}, \emph{\DUrole{n}{addr}}}{}
\sphinxAtStartPar
This is the thread which receives messages from the client and responds accordingly.
Messages are sent from the server to the clients connected to that particular server.
The following codes are received by the server which have the following meaning:
\begin{quote}

\begin{DUlineblock}{0em}
\item[] 1. cg: Check if the group already exists or not
\item[] 2. ci: Check if the individual exists or not
\item[] 3. ng: Create a new group with the client sending the code as admin
\item[] 4. eg: Check if the group exists and the client sending the code to the server is the admin
\item[] 5. ai: Add an individual to the group
\item[] 6. ri: Remove an individual from the group
\item[] 7. sa: Send the names of all the members of the group to the admin of the group
\item[] 8. wg: Send the text message in encrypted format to an individual
\item[] 9. ig: Send an image in an encrypted format to a group
\item[] 10. wi: Send the text message in encrypted format to an individual
\item[] 11. ii: Send an image in encrypted format to an individual
\end{DUlineblock}
\end{quote}
\begin{quote}\begin{description}
\item[{Parameters}] \leavevmode\begin{itemize}
\item {} 
\sphinxAtStartPar
\sphinxstyleliteralstrong{\sphinxupquote{conn}} (\sphinxstyleliteralemphasis{\sphinxupquote{socket.socket}}) \textendash{} It is the connection object of the client which has established a socket connection with the server

\item {} 
\sphinxAtStartPar
\sphinxstyleliteralstrong{\sphinxupquote{addr}} (\sphinxstyleliteralemphasis{\sphinxupquote{tuple}}) \textendash{} It is the address (ip,port) of the client which has established a socket connection with the server

\end{itemize}

\end{description}\end{quote}

\end{fulllineitems}

\index{letsconnect() (in module server)@\spxentry{letsconnect()}\spxextra{in module server}}

\begin{fulllineitems}
\phantomsection\label{\detokenize{server:server.letsconnect}}\pysiglinewithargsret{\sphinxcode{\sphinxupquote{server.}}\sphinxbfcode{\sphinxupquote{letsconnect}}}{\emph{\DUrole{n}{ip}}, \emph{\DUrole{n}{port}}}{}
\sphinxAtStartPar
This function establishes socket connection between the server to the server with the ip and port passed as parameter
The following codes are received by the server which have the following meanings:
\begin{quote}

\begin{DUlineblock}{0em}
\item[] 1. wi: Send the text message in encrypted format to the server to which the receiver client is connected
\item[] 2. ii: Send the image in encrypted format to the server to which the receiver client is connected
\item[] 3. wg: Send the text message sent by a client in a group in encrypted format to the server to which the receiver client (group member) is connected
\item[] 4. ig: Send the image sent by a client in a group in encrypted format to the server to which the receiver client (group member) is connected
\item[] 5. gk: Get the encryoted group private key from the loadbalancer
\end{DUlineblock}
\end{quote}
\begin{quote}\begin{description}
\item[{Parameters}] \leavevmode\begin{itemize}
\item {} 
\sphinxAtStartPar
\sphinxstyleliteralstrong{\sphinxupquote{ip}} (\sphinxstyleliteralemphasis{\sphinxupquote{str}}) \textendash{} It is the ip address of the server which wants to establish a socket connection with this server

\item {} 
\sphinxAtStartPar
\sphinxstyleliteralstrong{\sphinxupquote{port}} (\sphinxstyleliteralemphasis{\sphinxupquote{str}}) \textendash{} It is the port number of the server which wants to establish a socket connection with this server

\end{itemize}

\end{description}\end{quote}

\end{fulllineitems}

\index{main() (in module server)@\spxentry{main()}\spxextra{in module server}}

\begin{fulllineitems}
\phantomsection\label{\detokenize{server:server.main}}\pysiglinewithargsret{\sphinxcode{\sphinxupquote{server.}}\sphinxbfcode{\sphinxupquote{main}}}{}{}
\sphinxAtStartPar
Accepts a connection request and stores two parameters,
conn which is a socket object for that user, and addr
which contains the IP address of the client that just
connected
It also initiates the client thread

\end{fulllineitems}

\index{remove() (in module server)@\spxentry{remove()}\spxextra{in module server}}

\begin{fulllineitems}
\phantomsection\label{\detokenize{server:server.remove}}\pysiglinewithargsret{\sphinxcode{\sphinxupquote{server.}}\sphinxbfcode{\sphinxupquote{remove}}}{\emph{\DUrole{n}{connection}}, \emph{\DUrole{n}{usr}}}{}
\sphinxAtStartPar
Removes the connection of the user with the server and loadbalancer that is change the status of the user from online to offline
The entry in the dictionary with the key as the username of the user is deleted from both the server to which the client is connected and the loadbalancer
\begin{quote}\begin{description}
\item[{Parameters}] \leavevmode\begin{itemize}
\item {} 
\sphinxAtStartPar
\sphinxstyleliteralstrong{\sphinxupquote{connection}} (\sphinxstyleliteralemphasis{\sphinxupquote{socket.socket}}) \textendash{} Connection with the user which has logged out which needs to be removed

\item {} 
\sphinxAtStartPar
\sphinxstyleliteralstrong{\sphinxupquote{usr}} (\sphinxstyleliteralemphasis{\sphinxupquote{str}}) \textendash{} The username of the user who has logged out

\end{itemize}

\end{description}\end{quote}

\end{fulllineitems}



\chapter{Indices and tables}
\label{\detokenize{index:indices-and-tables}}\begin{itemize}
\item {} 
\sphinxAtStartPar
\DUrole{xref,std,std-ref}{genindex}

\item {} 
\sphinxAtStartPar
\DUrole{xref,std,std-ref}{modindex}

\item {} 
\sphinxAtStartPar
\DUrole{xref,std,std-ref}{search}

\end{itemize}


\renewcommand{\indexname}{Python Module Index}
\begin{sphinxtheindex}
\let\bigletter\sphinxstyleindexlettergroup
\bigletter{c}
\item\relax\sphinxstyleindexentry{client}\sphinxstyleindexpageref{client:\detokenize{module-client}}
\indexspace
\bigletter{l}
\item\relax\sphinxstyleindexentry{loadbalancer}\sphinxstyleindexpageref{loadbalancer:\detokenize{module-loadbalancer}}
\indexspace
\bigletter{s}
\item\relax\sphinxstyleindexentry{server}\sphinxstyleindexpageref{server:\detokenize{module-server}}
\end{sphinxtheindex}

\renewcommand{\indexname}{Index}
\printindex
\end{document}